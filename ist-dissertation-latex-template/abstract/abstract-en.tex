%!TEX root = ../dissertation.tex

\begin{otherlanguage}{english}
\begin{abstract}
% Set the page style to show the page number
\thispagestyle{plain}
\abstractEnglishPageNumber

Having robots interact with humans in domestic environments, while completing
household tasks, requires a whole new degree of autonomy and reasoning. Typically, 
these kind of environments may push the robot's capabilities (i.e. robot has to pick an 
object on an unreachable location). Likewise, some actions performed
by the robot agent have a higher success rate due to its domain knowledge.
Cooperation between humans and robots can overcome these issues, resulting in
a higher amount of possible tasks for the robot. This is an example of Symbiotic
Autonomy.

In this scenario, at some point the robot has the option to
do some action by himself or to ask for help in case it's better for both agents
- in the short or long-term. 

The proposed approach uses a planning framework based on probabilistic logic programming,
the HYPE planner, which gathers at every step, observations from the environment, generating 
a grounded Markov Decision Process problem from the described domain and deciding the action he
should take in order to maximize its performance on this environment.
Furthermore, this architecture is benchmarked on a simulation environment from generated
observations as well as in a house with a robot and real human agents.

% Keywords
\begin{flushleft}

\keywords{Symbiotic Autonomy, Planning Under Uncertainty, Probabilistic Logic
Programming}

\end{flushleft}

\end{abstract}
\end{otherlanguage}
